\chapter{General Discussion}
\begin{center}
    \textit{“Although the study of genotypic effects of mutation has now been relocated to the well-lit \\ upper floors of biology, mutation’s phenotypic effect is still in  the “dingy basement” \\ where Hermann Muller found it in 1921.”} \citet{Houl13}
\end{center}

\section{Overview}
Almost forty years ago, \citet{Ture84} emphasised the need for better parameter estimates to resolve the maintenance of quantitative genetic variation, an appeal reiterated (in $\sim20$ year intervals) by \citet{John05}, and then again by \citet{Wals18c28}. Fundamentally, new genetic variation originates through mutation, with both selection and random genetic drift acting to determine how long a mutation persists in the population \citep{Land75,Kond92, John05, Zhan05}. Despite the ubiquity of genetic variation for quantitative traits \citep{Lync98c7}, mutation-selection balance (MSB) theory still fails to produce compelling explanations for the levels of heritability in natural populations \citep{Wals18c28}. Models still cannot simultaneously account for observed selection strength against new mutations with sensible amounts of genetic variance \citep[thoroughly reviewed by][]{Wals18c28}. Yet precise modelling requires reliable data, where current estimates of the strength of selection and the properties of mutation remain ambiguous, and observations appear ‘mutually incompatible’ \citep{John05}. Standardised estimates of new phenotypic variation due to mutations (mutational variance, $V_M$) spanning two orders of magnitude is a case in point \citep{Houl96,Lync99,Hall09}. And, although $V_M$ estimates comprised a large body of work across various taxa and trait types; they are a univariate measure, which reflects the historical framework that the maintenance of variation has been set in. Models that account for pleiotropic genetic associations are gaining traction \citep{Bart90,John05} but are then limited due to the paucity in multivariate estimates of mutation ($V_M$ and covariance for several traits). To further complicate matters, evolutionary forces cause a negative association between mutant effect and frequency, assuming the majority of new alleles are deleterious, alleles in MSB should be at low frequency. This genetic architecture presents challenges to biologically relevant characterisations of mutational effects. Therefore, more empirical work and amended experimental design is required that both addresses aspects of the reliability of estimates, such as $V_M$ and the multivariate $\vec{M}$ matrix, whilst affording the opportunity to utilise genomic data to explore trait genetic architecture.\par

Since its conception, \citet{Muka64}’s mutation accumulation (MA) design to investigate mildly deleterious mutations has been implemented in numerous different taxa, investigating various individual traits \citep{Houl96, Hall09, Wals18c12}. The MA design has been foundational in addressing questions concerning the rate, magnitude of effect and nature of new spontaneous mutations. With the increasing accessibility of genome sequencing, we can only expect to see more versatile implementation of the MA design, particularly in directly understanding the genomic architecture of mutations when coupled with whole-genome sequencing \citep{Katj19}. In this thesis, I have modified the MA design to relax selection after twenty generations of mutation accumulation. In Chapter 2, I used this modified design in conjunction with a meta-analysis to address biological and experimental factors that could contribute variability to mutational variance estimates. In Chapter 3, I expand on my findings from Chapter 2, and investigate my wing traits in a multivariate framework to characterise general patterns of pleiotropic effects of mutations on functionally related traits. In Chapter 4, I again utilised the paired subline design in combination with whole-genome sequencing, to identify patterns of selection for mutant alleles that potentially contributed to an observation of exceptionally high fitness. Below I discuss the major findings of each of the Chapters taking into consideration how my work attempts to improve detection of mutational effects whilst building on existing comprehension of the mutation-selection models. \par

\section{Heterogeneity in mutational variance estimates}
Estimates of $V_M$ have previously been documented and investigated, where in both of their respective reviews, \citet{Lync99} and \citet{Houl96} remark on the contribution that sampling error potentially plays in estimate variability. Yet despite the comprehensive body of mutational variance estimates, there is insufficient replication within a trait or taxon under the same experimental conditions, where authors can only conclude of the generality of results. When considering the average magnitude of standardised estimates, factors such as taxon \citep{Lync99, Hall09} and trait type \citep{Houl96, Houl98} were implicated as causal factors. In Chapter 2, I build on previous tests by implementing a meta-analysis on the updated set of published estimates. My investigation provided some support for a difference among trait types in the magnitude of mutational variance, but not taxon. I also intended to examine how the duration of mutation accumulation affected estimates \citep{Lync86,Mack95}, however, the number of MA generations was confounded with taxon, precluding conclusions about heterogeneity in rate over time. Generally, the meta-analysis revealed substantial confoundment between potential causal factors. I therefore utilised my modified MA design, where I repeatedly estimated the among-line (mutational) variance to investigate whether unintended environmental heterogeneity, or transient within-line segregation of mutations can contribute variation among standardised estimates of $V_M$. The results from this experiment suggest that environmental conditions or sampling transient segregating mutations can contribute to variability in $V_M$, however, these effects were infrequently detected. I concluded that sampling error was responsible for variability between point estimates. \par

Typically, sampling error in well design quantitative breeding designs is minor \citep{Falc96}, however, the magnitude of mutational variance is a comparably tiny component of phenotypic variance, and can easily be swamped by measurement error. Sampling error is likely to be both inflating and deflating estimates, where both will consequently bias mutation-selection theory. More estimates of $V_M$ might uncover the error distribution about the true mean. However, it is worth noting that in our meta-analysis we observed: study bias, where there was over-representation for flies and nematodes; and, poor data reporting or variable study focus meant standard errors or components of variance were often omitted, resulting in a reduced number of comparable studies (and hence independent estimates of sampling error). Future endeavours should therefore focus on diminishing the effect of sampling error. Usually error is mitigated through attainment of large numbers of samples from the level of biological interest \citep[e.g., including more lines,][]{Lync88a}, however the logistically prohibitive nature of mutation accumulation experiments makes increasing sample size difficult. My results from Chapter 2, suggest that by using repeated measures of the traits from consecutive generations might considerably improve estimates, without the need to maintain larger numbers of mutation lines for many generations. To deepen our understanding of whether mutational variance does vary in magnitude among traits, taxa and over time, further robust and directly comparable estimates are needed. \par

\section{Accounting for pleiotropic mutation associations}
In Chapter 2, an aspect of my approach to consider is that I treated each wing trait as independent, assuming no correlation across traits in the effects of mutation or selection. This assumption is unlikely to be strictly met as mutational pleiotropy is expected to simultaneously affect these traits to some extent. The effect of non-independence on estimates of selection is not clear, and depends on the unknown distribution of effects of pleiotropic mutations on each trait and fitness \citep{Paab13}. Pleiotropic mutations are predicted to be under stronger selection than mutations with effects only on individual traits \citep{Fish30,Este06,Wagn08,McGu14}. Empirical studies of the nature and extent of mutational pleiotropy remains severely lacking, and subsequently theoretical models cannot robustly incorporate the phenomenon \citep{John05}. 

In Chapter 3, I expand on my analysis in Chapter 2 by considering covariance between six functionally related \textit{Drosophila serrata} wing shape traits. The major finding of this chapter was that there was strong covariance between wing traits, which resulted in a single trait combination that was consistently associated with mutational variance ($m_{max}$). Thus, univariate estimates of mutation variance are likely overestimated. My observation of reduced dimensionality in the mutational space is consistent with other studies in \textit{Drosophilid} wings \citep{Houl13,Houl17,Duga21}, and suggest that there are trait combinations for which there is little to no mutational variance. These nearly null spaces potentially highlight constraints to evolution \citep{Gomu09, Houl13,Hine14}. Reduced dimensionality in mutational space is consistent with the existence of a limit to the number of independent traits under selection, where strong selection operates on only a few dimensions of multivariate trait space \citep{Bart90, Wals09}, where many traits will experience a weak effect of selection. For example, major MSB models cannot account for strong selection and overestimate the amount of standing genetic variance in univariate estimates of mutational variance in \textit{Drosophila wings} \citep{McGu15}, but, when considering wings in a multivariate context, \citep{Duga21} estimated much stronger selection, which potential could reconcile MSB differences. \par

In my research, I utilised wings as an easy-to-acquire, multi-trait, phenotype that can aid in reducing the cumbersome nature of measuring phenotypes. There is a rich literature behind wings, including studies of sexual selection \citep{McGu11a, Carr11} and sexual antagonism \citep{Szte19}, clinal adaptation \citep{Hoff02} and multivariate evolution \citep{Houl17}. However, differences among studies in the specific measures used to characterise wing shape do limit comparability, and inference of whether mutation rates might vary. Due to high multicollinearity \citep{McGu15,Szte15, Duga21}, the wing landmarks that I use in this thesis differ from sets of ILDs previously used to describe wing shape in \textit{D. serrata}\citep{McGu15,Szte15, Duga21,McGu13}. In Appendix 4, I outlined how I reduced my wing shape trait set from 36 unique combinations of distances calculated between nine wing vein intersections to the ten analysed in Chapter 2 (of which five were in Chapter 5), chosen to capture independent variation that spanned the total wing. Larger sample sizes will facilitate the inclusion of more wing dimensions and will be comparable to the 23 dimensions used by Houle \textit{et al.} \citeyear{Houl13,Houl17}, although the alignment of landmarks, and subsequent definition of relative warps to account for the loss of dimensions due to alignment may always contribute differences among studies. Inconsistently detecting mutational variance in multiple univariate wing shape traits across studies is consistent with my conclusion in Chapter 2 that sampling error is responsible for low repeatability in $V_M$. However, when accounting for mutational pleiotropy (covariance) between wing traits, I increased the statistical power of detection of mutational variance and was able to detect biologically relevant changes in allometry that were similarly observed in other studies. \par

\section{Rare alleles contribute to the maintenance of quantitative genetic variance}
In Chapter 2 and 3, we have assumed that observed mutations are unconditionally deleterious. This assumption is consistent with the long-standing inference of purifying selection on coding DNA \citep{Kimu83, Ohta92,Li97} and fitness declines under mutation accumulation \citep{Hall09}. Beneficial mutations have been observed, particularly in the experimental evolution of micro-organisms
\citep[e.g.,][]{Jose04}, but also in MA experiments \citep[e.g.,][]{Shaw00}, although prevailing evidence suggests that beneficial mutations are rare \citep{Bata14, Heil14, Rock15}. Failure to account for rare, weak-effect beneficial mutations might inflate the inferred frequency of weakly deleterious mutations \citep{Tata17}, biasing the inferred distribution of deleterious effects to be more leptokurtic than the true distribution. Also, different MSB models assume different frequency distributions mutations, for example, Gaussian models assume an abundance of small effect mutations, whereas house-of-cards models assume rarer large effect mutations \citep{Land75, Ture84, Ture85, Wals18c28}. In Chapter 4, I took advantage of the sublines maintained at two different population sizes, small ($N =24$) and large ($N = 288$), for seven generations; altering the threshold that segregating mutations were visible to selection ($s = 1/2N_e$). I used whole-genome sequencing on pooled flies sampled separately for each subline, in each sequential generation, to identify patterns of genomic sweeps. Mutation accumulation and whole-genome sequencing has previously been used to infer the genomic rate and effect of deleterious mutations (\citealp[e.g.,][]{Keig09a, Huan16} \citealp[reviewed by][]{Katj19}). 

In line with emerging evidence that phenotypic effects are the results of rare large-effect mutations \citep{Mack92A, Davi99, Heil14, McGu14b}, I expected to detect a single large (beneficial) effect locus that accounted for the extreme mating success observed in the large population for a single line. Although I was unsuccessful in identifying a single large effect causal locus, the major finding of Chapter 4 was identifying the signature of mutation-selection-balance against rare, weak effect ($\sim 0.003 < s < 0.029$) variants. Both observations from the nucleotide diversity estimates and changing allele frequencies suggest that the smaller population had an increased mutational load. Furthermore, when considering potential pleiotropy amongst the multiple independent regions in the genome, there was substantial gene ontology enrichment for these relatively few (58) loci. Currently, theoretical models differ in the extent of pleiotropy among traits, and the magnitude of selection against pleiotropic effects \citep{Ture85, Bart90,John05}. With some slight modifications to the experimental design (outlined below), there is exciting potential in replicating my experiment to address the mechanisms of pleiotropy in rare deleterious loci and how this contributes to MSB theory. \par


\section{Future directions}
\subsection{Applying our design to measure variability in fitness traits}

The majority of MSB models require estimates of fitness to estimate the strength of selection \citep{Wals18c28}.   Unlike morphological traits (such as wing shape), 
life-history traits (proxy measures of total fitness) are expected to have larger mutational targets and greater sensitivity to environmental perturbations \citep{Houl96, Houl98},  and consequently are inherently much noisier traits \citep{Lync98c12,Wals18c28}.  I would recommend the application of my repeated-estimates MA design in determining the repeatability of $V_M$ estimates for fitness (life history) traits, particularly in a multivariate framework. Currently, the available body of M-matrix estimates for fitness traits are likely biased due to the linkage effects of accumulated deleterious mutant alleles, obscuring the nature of pleiotropy. It is likely that through repeated measures, more nuanced trait combinations will be determined. \par

\subsection{Modification of our experimental design in future genomic analyses}
As outlined in Chapter 2, my experimental design has great potential in future applications but with some changes to the design. I would recommend 1) applying the population manipulation longer, for at least $0.1N_e$ generations, to increase chances of detecting a sweep \citep{Wals18c9}; 2) sampling phenotypes concurrently with genomes which would facilitate more direct targeting of causal loci and genomic regions under selection; 3) although I met recommendations for the number of individuals sampled for detecting strongly selected loci \citep[$50N$,][]{Schl15}, I would recommend at least $200N$ to increase chances of detecting weak effect loci; and 4) when imputing allele frequencies, to favour models that account for both heterogeneity introduced through genome sampling and demographic stochasticity.\par

\subsection{Non-additive effects}
Contrary to classical MA, our MA lines ranged from 24 to 288 individuals, suggesting that the majority of new mutations arising over the five (wings; DNA, seven) generations of this study were present in heterozygotes. Therefore, assays of divergence between the paired lines over time are relatively representative of natural, randomly mating populations, where selection operates on heterozygous fitness \citep{Simm77}. Although many classical MA experiments have employed crosses to assess mutational effects in heterozygous form, the distribution of dominance coefficients of mutations remains poorly resolved \citep{Lync98c12, Hall09, Agra11}. Further investigation is required to determine the extent to which dominance influences mutational estimates from finite populations and might contribute to the discrepancy between DNA sequence and MA estimates of the distribution of mutational effects. Applying population size treatments in MA designs allow the threshold of neutrality to be manipulated, where the dominance coefficients at different opportunities for selection can be investigated. \par

\section{Conclusion}
In this thesis, I have presented a broadly applicable approach that I used to directly target hypotheses about estimation and the nature of new mutations.  I demonstrated that sampling error is pervasive in univariate estimates of mutational variance and potentially biases our current understanding of mutation-selection balance.  When combining my approach with a multivariate analysis I show that there are two benefits; firstly, by accounting for pleiotropy in functionally related traits detecting mutational variance was improved, and secondly, the nature of pleiotropy of new mutations could be explored.  Finally,  I paired my approach with genomic analysis and found evidence of increased selection against rare variants. Further empirical work is required to explore the generality of the conclusions that I have made and validate the robustness of my approach in improving the detection of mutational variance in fitness traits,  investigating the contribution of rare variants and the extent of pleiotropy.  