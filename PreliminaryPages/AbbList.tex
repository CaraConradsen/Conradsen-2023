\chapter{List of Abbreviations and Symbols}

\begin{table}[htp]
\begin{center}
\renewcommand{\arraystretch}{1.3} % Default value: 1
\begin{tabular}{cl}
\multicolumn{2}{c}{\MakeUppercase{Generally used Abbreviations and Symbols}}\\
&\\
MA & Mutation accumulation\\
MSB & Mutation selection balance\\
ML & Maximum likelihood \\
REML & Restricted maximum likelihood \\
$\hat\theta$ & Vector of REML random effect parameter estimates\\
$\vec{V}$ & The inverse of the Fisher information matrix, from restricted maximum \\[-0.8ex]
& likelihood mixed models\\
REML-MVN & REML-multivariate normal distribution sampling approach, which approximates\\[-0.8ex]
& the sampling error of genetic covariances (Meyer and Houle, \citeyear{Meye13,Houl15})\\
LRT & Likelihood ratio test \\
FDR & False discovery rate \\
$V_L$ &  Among-line variance\\
$V_G$ &  Additive genetic variance\\
$V_M$ &  Mutational variance \\
$V_E$ &  Environmental variance\\
$CV_M$ & Coefficient of mutational variance\\
$h_M^2$ & Mutational heritability\\
$s$ & Selection coefficient\\
$N_e$ & Effective population size\\
$L$ & Large population size treatment, maintained as 288 flies per generation\\
$S$ & Small population size treatment, maintained as 24 flies per generation\\
$\Bar{X}$ & Trait mean \\
G $\times$ E & Genotype by environment interaction \\
BLUPs & Best linear unbiased predictors \\
DsRGP & \textit{Drosophila serrata} Reference Genome Panel \\
ILD & Inter-landmark distance, the distance between wing vein landmark positions\\
CS & Centroid size\\

%=================== PAGE BREAK ============================%
\end{tabular}
\end{center}
\end{table}
%%%%%%%%%%%%%%%%%%%%%%%%%%%
\begin{table}[htp]
\begin{center}
\renewcommand{\arraystretch}{1.3} % Default value: 1
\begin{tabular}{cl}
%=================== PAGE BREAK ============================%
% SECOND SECTION
&\\
\multicolumn{2}{c}{\MakeUppercase{Chapter Specific Abbreviations and Symbols}}\\
% Chapter 2
&\\
\multicolumn{2}{l}{\textbf{Chapter 2}}\\
\midrule
cv & Coefficient of variance used to compare variability among published estimates\\
% Chapter 3
&\\
\multicolumn{2}{l}{\textbf{Chapter 3}}\\
\midrule
$\vec{M}$ & The mutational covariance matrix; measured as the \textit{among-line} (mutational) variance \\
$m_{max}$ & Major axis of mutational variance \\
$\lambda$ & Eigenvalue \\
$\vec{R}$ & The \textit{residual} covariance matrix \\
$\vec{H}$ & Krzanowski’s subspaces matrix\\
$h_k$ & The $k^{th}$ eigenvector of Krzanowski’s subspace \\
$\vec{L}$ & Matrix containing the subset of the largest eigenvectors\\
$\vec{D}$ & Symmetrical relative distance matrix, using \citet{Mitt09}’s \\[-0.8ex]
& relative genetic distance\\
\textbf{$\Sigma$}$_{\vec{M}}$ & Fourth-order covariance tensor of the \textit{among-line} covariance matrices\\
\textbf{$\Sigma$}$_{\vec{R}}$ & Fourth-order covariance tensor of the \textit{residual} covariance matrices\\
$\vec{S}$ & The second-order symmetric matrix, which \textbf{$\Sigma$}$_{\vec{M}}$ (or \textbf{$\Sigma$}$_{\vec{R}}$) was mapped onto\\ 
$\vec{E}_k$ & The $k^{th}$ eigentensor for \textbf{$\Sigma$}$_{\vec{M}}$ or \textbf{$\Sigma$}$_{\vec{R}}$. \\
$e_{k,i}$ & The $i^{th}$ eigenvector for the $k^{th}$ eigentensor\\
$C^{p,k}$ & Coordinate of the $k^{th}$ eigentensor for the $p^{th}$ covariance matrix\\
% Chapter 4
&\\
\multicolumn{2}{l}{\textbf{Chapter 4}}\\
\midrule
$p$ & Major allele frequency\\
$q$ & Minor allele frequency \\
MAF & Minor allele frequency \\
$\Delta p$ & Major allele frequency difference between the large and small population size treatment \\
$\rho\Delta p$ & Spearman's rank coefficient for allele frequency difference between two treatments \\
$F_{ST}$ & Fixation index, a measure of population differentiation\\
$\pi$ & Tajima’s nucleotide diversity \\
Indel & Small insertions and deletions\\
SNP & Single nucleotide polymorphism \\
WGS & Whole genome sequencing \\
GO & Gene ontology\\ 
E\&R & Evolve and resequencing experimental design \\
\end{tabular}
\end{center}
\end{table}
