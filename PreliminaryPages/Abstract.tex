\noindent\textbf{\underline{Abstract}}\\

\noindent Mutation has a paradoxical role in evolution: it is the ultimate source of new genetic variation which enables adaption in natural populations, yet most mutations have an adverse effect on fitness. How mutation in the presence of selection contributes to standing genetic variation is a fundamental concept to evolutionary biology, encapsulated best by the mutation-selection balance class of theoretical models. The magnitude of effect that a mutation has on fitness will determine how strongly natural selection will act to remove it from the population, where mutations of large effect are purged quickly, and weak mutations can persist for hundreds of generations, evolving under stochastic drift. Despite their importance, the fitness effects of mutations remain poorly understood. Mutation-selection balance models fail to reconcile empirical estimates of mutation, standing genetic variation and strength of selection, which may reflect our limited knowledge of the distribution of mutational effects.\par

In this thesis, using a mutation accumulation design modified to manipulate mutation-selection-drift dynamics, I investigate hypotheses regarding the estimation and nature of new mutations in the Australian vinegar fly, \textit{Drosophila serrata}. Understanding the degree to which estimates of mutation reflect true values might generate more realistic assumptions and resolve the application of mutation-selection balance models. In Chapter 2, I investigated heterogeneity in empirical estimates of new phenotypic variation introduced via mutation using two complementary approaches. I first employed a meta-analysis to discern if differences in taxa or trait type accounted for the two orders of difference reported in standardised estimates. There was equivocal support for estimates varying due to trait type assayed (life history versus morphology), but various experimental factors were confounded among the analysed studies, precluding further inquiry. Then, using data from the modified mutation accumulation experiment, where wing traits were measured for six consecutive generations under the different opportunities for selection, I specifically addressed the contribution of unintentional experimentally induced heterogeneity or random sampling of segregating mutations. The results suggested that sampling error contributed substantial variation, and most likely accounts for a substantial portion of the differences among published estimates.\par

A new mutation can be involved in multiple biological processes (pleiotropy), contributing genetic variation to multiple phenotypes. Empirical and theoretical attention has focused on the extent to which mutational pleiotropy may constrain evolution. However, if utilising pleiotropy within the experimental design may provide a method of improving estimation of mutational effects. In Chapter~3, I explore mutationally induced associations between six functionally related \textit{Drosophila serrata} wing shape traits. Employing higher order analytical frameworks, I found that the major axis of mutational variance, $m_{max}$, was consistently detected in all population samples. This observation contrasted with results from the study of individual traits in Chapter 2, and demonstrated that utilising pleiotropic covariance could improve estimation of mutational effects. Contrary to expectation, temporal divergence was modest and did not occur along $m_{max}$, suggesting evolution of pleiotropic mutational effect is restricted. \par

The consensus in both theoretical work and empirical observations is that the majority of new mutations are detrimental to fitness, yet adaptive evolution relies on the presence of rare beneficial mutations. In Chapter 4, I examined competitive mating success for my modified mutation accumulation lines after seven generations of increased opportunity for selection to act. A single serendipitous observation of exceptionally high fitness in a subline motivated further investigation. Exploiting my paired subline design to access the putative ancestral genome, and using whole-genome sequencing of the seven sequential generations during which the mutation must have arisen, I aimed to identify the locus that provided the putative beneficial mutation. I found evidence of increased selection against multiple deleterious variants with small effect sizes however I was unsuccessful in detecting a single large-effect causal locus to explain the divergence in fitness. Overall, the work in this thesis builds on existing comprehension of the mutation-selection models and attempts to improve detection of mutational effects whilst illuminating the genomic architecture underlying new mutations. \par

