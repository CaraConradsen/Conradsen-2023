\nobibliography* % suppresses bibliography at the end of the document

\urlstyle{same}
\begin{landscape}
\begin{table}[!ht]
\caption[Information on mutational variance estimates obtained from the literature.]{\textbf{Information on mutational variance estimates obtained from the literature.} We detail estimates that were included in the meta-analysis (A) and those that were disqualified (B, C). For each paper, we detail the source of the estimates of scaled mutational variance (mutational heritability, $h_M^2$ and, coefficient of mutational variance, $CV_M$) and associated error estimates\textsuperscript{\textdagger}. Studies (or individual estimates within studies) with scaled estimate (and error) required for the meta-analysis, but which were excluded on based on some additional data criteria are listed (and exclusion justified) in (B). In C, we list papers (estimates) that were excluded due to insufficient information to obtain standard error estimates for the scaled mutational parameters. NB: Some papers appear in (A) and (B) or (C). Only estimates identified in (A) were included in the meta-analysis. Estimates are available at:  \url{https://figshare.com/articles/dataset/MA_dserrata_wings_csv/14913051}.}
\label{tab:TabS1}
\scriptsize
\singlespacing
% \renewcommand{\arraystretch}{1.25}
\vspace{-1.5em}
\begin{tabular}{>{\hangindent=2em}p{4.7in}>{\vspace{0em}}p{4.7in}}
  \toprule
Reference & {\vspace{-1em}}Description of where estimates were obtained from \\ 
  \midrule
& \\[-1.5ex]
 \multicolumn{2}{c}{\textbf{\MakeUppercase{(A) Estimates used in the Meta-analysis}}}\\ 
& \\[-2ex]
\bibentry{Avil02} & $CV_M$ from Halligan and Keightley$^1$ Table 1.\\
\bibentry{Azev02} & Table 1 of paper with scale of mean corrected based on Figure 1.\\
\bibentry{Baer05} & $h^2_M$, $V_M$ and $CV_M$ estimates calculated from information in paper’s Table 1 and Supplementary Tables 3, 4 and 5. Error estimated by sampling\textsuperscript{\textdagger}. \\
\bibentry{Baer10a} & Table 2 of paper; $V_M$ reported as $I = \frac{V_M}{\bar{X}^2}$ and therefore $CV_M$ calculated as: $100\times\sqrt{I}$. Confidence intervals from bootstrapping where parameter constrained to be positive. \\
\bibentry{Caba02} & $CV_M$ calculated from in-text information, taking $V_B$ from ${\Delta}V=V_B$ / ($mean_c \times F_c$); where $F_c=203$ and $mean_c = 0.625$. Error estimated by sampling\textsuperscript{\textdagger}.\\
\bibentry{Char04} & $CV_M$ (and error) estimates taken from Halligan and Keightley$^1$ Table 1. \\
\bibentry{Clar95} & Trait means were taken from Table 1. $h^2_M$ and $V_M$ estimates taken from Table 4; $V_M$ was multiplied by 2.5 (scale for whole genome, as stated by authors) before calculating $CV_M$. Error in $CV_M$ estimated by sampling\textsuperscript{\textdagger}. \\
\bibentry{Davi16} & Table S1 of paper. Note: No estimate of $CV_M$ for citric acid as mean mis-reported in Table S1 and correct value unknown.  Confidence intervals from bootstrapping where parameter constrained to be positive. \\
\bibentry{Este05} & We calculated $CV_M$ using $V_M$ (Table 1) and the mean from Table 1 of the companion paper by Ajie \textit{et al.}$^3$. $h2_M$ was from Table 1 of Ajie \textit{et al.}$^3$  \\
\bibentry{Etie15} & Table 1 of paper.  Error in $h^2_M$ estimated by sampling\textsuperscript{\textdagger}.  Confidence intervals from bootstrapping where parameter constrained to be positive. \\
% ========Gap =======&
\bottomrule
\end{tabular}
\end{table}
\begin{table}[!ht]
\vspace{-0.5em}
\scriptsize
\singlespacing
\renewcommand{\arraystretch}{0.9}
\begin{tabular}{>{\hangindent=2em}p{4.7in}>{\vspace{0em}}p{4.7in}}
\midrule
% ========Gap =======&
\bibentry{Fern96} & $V_M$ and $h^2_M$ taken from Table 3 and trait mean from Table 1 of paper; $CV_M$ calculated from those estimates (but no error estimate available). \\
\bibentry{Fry02} & Trait means were from Table 1 and mutational variances were from Table 3; $CV_M$ was calculated from those estimates. Error estimated by sampling\textsuperscript{\textdagger}.  \\
\bibentry{Fry99} & Trait means and mutational variances were from Fry and Heinsohn (2002)’s Table 1 and Table 3, respectively, where $CV_M$ was calculated from those estimates.  \\
\bibentry{Garc00} & Table 2 of paper. \\
\bibentry{Hara95} & $CV_M$ was calculated from information in Table 3 and 4 of paper; $V_M$ was calculated using $4V_G$ (reported) and divided by $2\times$ the number of generations. Error estimated by sampling\textsuperscript{\textdagger}.  \\
\bibentry{Houl94} & $V_M$ from Table 4 of paper ($\times 2.5$ to scale to genomewide as indicated by the authors); data was standardised to mean=1 prior to estimation, and $CV_M = 100*\sqrt{V_M}$. Error estimated by sampling\textsuperscript{\textdagger}. \\
\bibentry{Houl04} & Table 3 of paper (for IVe-39 sublines only, constrained intercept), where $CV_M = 100*\sqrt{\Delta{V}}$ as described in text of paper on page 9. Error estimated by sampling\textsuperscript{\textdagger}. \\
\bibentry{Joyn09} & Table 1 and 2 of paper.\\
\bibentry{Keig97} & $h^2_M$ estimate (and SE) provided in text on pg 3825. Estimates for productivity determined from among-line variances and trait mean taken from Figure 2 using the R package \textit{metaDigitise}. Error in $CV_M$ estimated by sampling\textsuperscript{\textdagger}. \\
\bibentry{Lati14} & Table 1 of paper \\
\bibentry{Lope93} & We used estimates from B (full-sib mating) lines only, where trait means were from figure 1 (males=30, females = 36.8). Among-line variance and $h^2_M$ for bristle number from Table 6, $CV_M$ calculated from the provided information, but no error estimate available.\\
\bibentry{Lync85} & $h^2_M$ from Table 1 and $CV_M$ calculated from information in Table 1 (but no SE provided for mean and hence unavailable for $CV_M$). Error estimated by sampling\textsuperscript{\textdagger}.\\
\bibentry{Lync98b} & Estimates of $h^2_M$, $V_M$, $V_E$ and trait mean (MA mean calculated for the final generation as: $z_0+R_m*30$) from Table 1 of paper and $CV_M$ calculated from this information. Error in $CV_M$ estimated by sampling\textsuperscript{\textdagger}: $z_0$ and $R_m$ were sampled to allow the mean to be determined per sample.\\
\bibentry{Mack92B} & Table 3 of paper, assuming $N = 0.7N_e$. Estimates available for per generation (Table 1), but SE not reported.\\
\bibentry{Mart98} & Table 2 of paper. \\
\bibentry{McGu11a} & $h^2_M$ was taken from text. $CV_M$ was calculated by determining $V_M$ as (mean ($V_E$) × $h^2_M$), where $V_E$ and MA means were taken from Tables S1 and S2 of paper for the Control MA population. Error estimated by sampling\textsuperscript{\textdagger}.\\
% ========Gap =======&
\bottomrule
\end{tabular}
\end{table}
\begin{table}[!ht]
\vspace{-0.em}
\scriptsize
\singlespacing
\renewcommand{\arraystretch}{0.95}
\begin{tabular}{>{\hangindent=2em}p{4.7in}>{\vspace{0em}}p{4.7in}}
\midrule
% ========Gap =======&
\bibentry{Muka72} & Among-line and mean estimates from Tables 1, 2 and 3 of paper for Generation 40 only were used to calculate $V_M$ and $CV_M$.\\
\bibentry{Ostr07} & Trait mean from Table 1 and $CV_M$ (MA mean), $V_M$ and $h^2_M$ from Table 2 of paper. While log transformation had little impact on scaled estimates, $V_M$ and $V_E$ were extreme on the raw scale, while mean of the log-transformed data was negative.\\
\bibentry{Plet98} & $h^2_M$, $V_M$ and mean from Table 5 of paper and $CV_M$ estimated from that information. Error estimated by sampling\textsuperscript{\textdagger}.\\
\bibentry{Role16} & Table 2 of paper. We included $CV_M$ for fruit number and total fitness however authors have calculated $CV_M$ as $100(e^{\sqrt{V_M}}-1)$ as per Poisson values. Parameter confidence intervals from MCMCglmm, and constrained to be positive. \\
\bibentry{Rutt10} & Table 2 of paper. \\
\bibentry{Sant92} & $h^2_M$ and $V_E$ from Table 4 for full-sib mating lines only (B lines; all lines). Trait means were taken from Figure 3 (sternopleural bristles =21.7, wing length = 1210 and wing width = 465). $CV_M$ estimated from that information but error estimates not available.\\
\bibentry{Scho05} & Trait means and mutational variances were taken from Table 1 and 3 for flower number and weight
respectively. $V_M$, $h^2_M$, and $CV_M$ were estimated from that information, where $V_L$ for MA was corrected for
initial (control variance) if that was non-zero; error estimates were similar between control and mutant and
MA SE were used to approximate error for meta-analysis.\\
\bibentry{Schu99} & From Table 3; $CV_M$ not available as calculated from among-line variance, not the per-generation $V_M$.\\
\bibentry{Shaw00} & Estimates of mean and among-line and environmental variances taken from Table 1 of paper (generation 17), and estimates of $V_M$, $h^2_M$, and $CV_M$ from Table 2. To estimate error for $h^2_M$, and $CV_M$, the among-line and mean estimates for generation 17 were used, along with $V_M$ from the joint analysis. Error estimated by sampling\textsuperscript{\textdagger}.\\
\bibentry{Vass00} & $h^2_M$ was taken from Table 2; trait mean (intercept) from Table 1 and $V_M$ ($slope/2$) from Table 2 were used to calculate $CV_M$. Error in $CV_M$ estimated by sampling\textsuperscript{\textdagger} following the approach detailed for Lynch \textit{et al.} (1998) above. For one trait (generation time) the reported standard error of the $V_M$ was 0, precluding calculation of errors. \\
\bibentry{Vass99} & $h^2_M$ and $V_M$ estimates were taken from Table 1 of paper. Error in $CV_M$ estimated by sampling\textsuperscript{\textdagger} following the approach detailed for Lynch \textit{et al.} (1998) above. For one trait (generation time) the estimated mean was $\sim0$, resulting in nonsensical $CV_M$ and hence this was excluded.\\
  \midrule
& \\[-1.5ex]
 \multicolumn{2}{c}{\textbf{\MakeUppercase{(B) ESTIMATES REMOVED FROM META-ANALYSIS DUE TO ISSUES WITH MUTATIONAL PARAMETER ESTIMATES}}}\\ 
& \\[-2ex]
\bibentry{Baer10a} & As in (A); Excluded \textit{Caenorhabditis brenneri} strain PB2801 $h^2_M$ and strain PB2802 $CV_M$ generation 100 estimates as they were outliers ($>3.5$ standard deviations from the mean and $>5$ times the inter-quartile range from the median).\\
\bibentry{Bail59} & From Table 1, and in-text information on number of generations; $h^2_m$ calculated as per Lynch $1988^4$. Excluded due to low representation of vertebrate taxa.\\
% ========Gap =======&
\bottomrule
\end{tabular}
\end{table}
\begin{table}[!ht]
\vspace{-0.5em}
\scriptsize
\singlespacing
% \renewcommand{\arraystretch}{1.25}
\begin{tabular}{>{\hangindent=2em}p{4.7in}>{\vspace{0em}}p{4.7in}}
\midrule
% ========Gap =======&
\bibentry{Case08} & MA litter mean was taken for generation 46 in Table 1. The $V_M$ and $h^2_m$ estimates were taken from text and Table 3. $CV_M$ was calculated from reported information. Excluded due to low representation of vertebrate taxa.\\
\bibentry{Clar95} & As in (A). ADH $h^2_m$ excluded as it was an extreme outlier ($>4$ standard deviations from the mean and $>5$ times the inter-quartile range from the median).\\
\bibentry{Down03} & $V_M$ and $h^2_m$ were taken from Table 1. Line means were estimated from figure 3 and used to calculate $CV_M$ (but not error). Excluded due to low representation of non-Drosophila insect taxa.\\
\bibentry{Farh16} & Trait means from Table 1 of paper, and variances from Table S4. One estimate (for posterior centrosome oscillation frequency) was removed as the estimate for $V_M$ was negative. Excluded due to low representation of trait type (mitotic cell division).\\
\bibentry{Hara95} & As in (A); Could not calculate $CV_M$ (negative mean) for one metabolite (Isocitrate-dehydrogenase activity) which was therefore excluded. Alpha-amalyse was excluded as the estimated SE of $CV_M$ was considerably larger than the estimate itself.\\
\bibentry{Houl94} & As in (A). Early male mating success excluded as $CV_M$ estimate was negative; late mating $CV_M$ excluded due extreme value ($>4$ standard deviations from the mean and $>5$ times the inter-quartile range from the median)\\
\bibentry{Joyn11} & Information in Table 1 of paper used to calculate $CV_M = 100*\sqrt{V_M}$, where $V_M$ (and $V_L$) were reported on intensity of selection scale, as described on pg 1441 of paper. We sampled from the distribution of parameters used to calculate $V_M$, but the standard error of the sample was orders of magnitude larger than the estimate. Therefore, this study was excluded.\\
\bibentry{Latt15} & Among-line variance (designated by authors as $\Delta$V) provided in Table S1 and used to calculated using $V_M = \Delta$V/(2*96), where 96 was the number of generations of MA reported by the authors on pg 245 of paper. $V_E$ and $\bar{X}$ were provided in Table S1 and, along with the calculated $V_M$, used to calculate $h^2_m$ and $CV_M$. However, the
reported confidence intervals for $\Delta$V are extremely large (infinite for several traits), and the standard errors calculated are always larger than the parameter estimate, and all estimates from this study were excluded. Notably, this study included only five MA lines, and poor quality estimates could be expected.\\
\bibentry{Lope93} & \\
\bibentry{Lync98} & As in (A). Second clutch size excluded as it was an extreme outlier ($>7$ standard deviations from the mean and $>5$ times the inter-quartile range from the median). Size at birth was excluded because calculation of $h^2_m$ from the data in Table 1 returned an estimate several orders of magnitude larger than reported, suggesting some incorrect parameter values reported. \\
\bibentry{Morg14} & $CV_M$ estimates were taken from Table 3 (second assay only, as one strain had missing data for the first assay), but no SE reported. To estimate SE it was necessary to use $V_L$ estimates for MA and ancestor from Table 3 (and reported CI), the generation number estimates from Table 2, and estimates of trait mean (CI) from Table 4. Excluded due to low representation of micro-organisms.\\
\bibentry{Pann08} & MA mean and line numbers taken from Table 2 of paper. Variance estimates were taken from Table 3, where $CV_M$ has been converted to a percentage scale. Excluded due to low representation of non-Drosophila insect taxa\\
% ========Gap =======&
\bottomrule
\end{tabular}
\end{table}
\begin{table}[!ht]
\vspace{-0.5em}
\scriptsize
\singlespacing
% \renewcommand{\arraystretch}{1.25}
\begin{tabular}{>{\hangindent=2em}p{4.7in}>{\vspace{0em}}p{4.7in}}
\midrule
% ========Gap =======&
  % \midrule
& \\[-1.5ex]
 \multicolumn{2}{c}{\textbf{\MakeUppercase{(C) ESTIMATES REMOVED FROM META-ANALYSIS DUE TO LACK OF INFORMATION ON ESTIMATION ERROR OF A MUTATIONAL PARAMETER}}}\\ 
& \\[-2ex]
\bibentry{Burc07} & $V_M$ was calculated from the log fitness variances reported on pg. 473, and MA mean was calculated as $1 - E(log W_1)$ to calculate $CV_M$.\\
\bibentry{Chan03} & Table 3 of paper.\\
\bibentry{Clar93} & Halligan \& Keightley$^1$ Table 1; no further detail in original paper to estimate SE.\\
\bibentry{Fest73} & Houle \textit{et al.} (1996)$^2$ Table 1; insufficient information in the original paper to determine error for scaled variance
estimates. \\
\bibentry{Gong05} & $CV_M$ estimates were taken from Halligan \& Keightley$^1$ Table 1, but were noted by. Halligan \& Keightley$^1$ as extreme outliers. Insufficient information to calculate errors for scaled estimates.\\
\bibentry{Hall08} & Table 2 of paper.\\
\bibentry{Huan16} & $h^2_m$ estimates were taken from supplementary information 1D but no estimate of error provided. \\
\bibentry{Jose04} & From in-text information on page 1819; NB: heritability estimate was incorrectly reported and corrected in Hall \textit{et al.} 2008 (citation details above) \\
\bibentry{Kava05} & Estimates from information in Table 5 of paper, but no errors reported in that table. Notably several estimates of mutational variance were negative (Table 5), and mutational variance was not statistically supported for any trait in this study.\\
\bibentry{Keig97} & Lifespan estimates were taken from text on p. 3826 of paper but no error estimate reported.\\
\bibentry{Kibo96} & Halligan \& Keightley$^1$ Table 1; no further detail in original paper to estimate SE.\\
\bibentry{Mack98} & Table 9 of paper: Only the MA maintenance temperature was recorded. No error estimates.\\
\bibentry{Muka64} & Houle \textit{et al.} (1996)$^2$ Table 1; $V_E$ complexity detailed in Houle \textit{et al.} (1996)$^2$ and lack of information on mean
error mean that no SE of scaled $V_M$ were obtained.\\
\bibentry{Muka68} & Table 1 of paper; no further detail in original paper to estimate SE of scaled $V_M$. \\
\bibentry{Muka84} & Table 1 and in text. Insufficient information to obtain error estimate for scaled estimate of $V_M$.\\
\bibentry{Ohni77} & Table 1 and in text (for zero level of mutagen only), but insufficient information to calculate SE for scaled estimates; Houle \textit{et al.} (1996)$^2$ Table 1.\\
% ========Gap =======&
\bottomrule
\end{tabular}
\end{table}
\begin{table}[!ht]
\vspace{-0.5em}
\scriptsize
\singlespacing
% \renewcommand{\arraystretch}{1.25}
\begin{tabular}{>{\hangindent=2em}p{4.7in}>{\vspace{0em}}p{4.7in}}
\midrule
% ========Gap =======&
\bibentry{Paxm57} & Houle \textit{et al.} (1996)$^2$ Table 1; no further detail in original paper to estimate SE.\\
\bibentry{Plet99} & Estimates of $V_M$ from Table 4 of paper, but no estimates of error of trait mean or $V_E$ reported and no scaled estimate with associated error derived. \\
\bibentry{Russ63} & Houle \textit{et al.} (1996)$^2$ Table 1; no further detail in original paper to estimate SE. The authors note (pp 177) that their genetic variation are “rather poorly estimated”. \\
\bibentry{Wayn98} & Table 3 of paper.\\
\bibentry{Xu04} & $h^2_m$ estimates from Table 2 of paper.\\
% ========Gap =======&
\bottomrule
\end{tabular}
\end{table}
\begin{table}[!ht]
\vspace{-19cm}
\tiny
\singlespacing
\renewcommand{\arraystretch}{2}
\begin{tabular}l
\textsuperscript{\textdagger} Where 95\% confidence intervals were reported, we converted to approximate standard errors as: (upper limit – lower limit) / 3.92. Where standard errors were not reported for $h^2_M$ or $CV_M$, but were reported for the parameters used to calculate these scaled mutational variance \\[-1em]
estimates (e.g., $V_M$, $V_E$, mean), as described in the Methods, we used the `rnorm` function in R to generate 10,000 random samples from the distribution defined by the estimate (mean) and standard error (variance), and calculated the scaled metric (for $h^2_M$ or $CV_M$) in each \\[-1em]
sample, then estimated the standard error.\\
$^1$ \bibentry{Hall09} \\
$^2$ \bibentry{Houl96}\\
$^3$ \bibentry{Ajie05} \\
$^4$ \bibentry{Lync88b} \\
% ========Gap =======&
\end{tabular}
\end{table}
\end{landscape}
\nobibliography{}
% \nobibliography{DFElib}

 

